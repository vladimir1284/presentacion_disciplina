\documentclass[aspectratio=43]{beamer}
% \documentclass[aspectratio=43]{beamer}
% ---------------------------
% Set the aspectration variable before!
% ---------------------------
\newcommand{\myaspectratio}{43} % <-- set to 43 or 169

% ---------------------------
% Custom colors
% ---------------------------
\definecolor{mainblue}{RGB}{100,149,237}
\definecolor{ghostwhite}{RGB}{242,244,245}

% ---------------------------
% Font (XeLaTeX or LuaLaTeX required)
% ---------------------------
%\usepackage{fontspec}
% Configure your custom font
% \setsansfont{AlanSans.ttf}[
%   Path=./AlanSansFont/,                      % Path to the folder containing your font files
%   Extension=.ttf,               % Specify the font file extension
%   UprightFont=*-UpRight.ttf,    % Link the regular font file
%   BoldFont=*-Regular.ttf           % Link the bold font file
%   % ItalicFont=*-Italic.ttf
% ]

% ---------------------------
% Emoji
% ---------------------------
% \usepackage{emoji} % LuaLatex
% \usepackage{twemojis} % XeLatex

% ---------------------------
% Social icons
% ---------------------------
% \usepackage{fontawesome5} 

% ---------------------------
% Minimal Beamer theme
% ---------------------------
\usetheme{default}
\setbeamercolor{normal text}{fg=black,bg=ghostwhite}
\setbeamercolor{structure}{fg=mainblue}
\setbeamercolor{frametitle}{fg=white,bg=}

% Move frametitle slightly down
\setbeamertemplate{frametitle}{%
  \vspace{0.5cm} % <-- adjust this value
  \nointerlineskip%
  \begin{beamercolorbox}[wd=\paperwidth,sep=0.3cm,left]{frametitle}%
  \usebeamerfont{frametitle}\insertframetitle
  \end{beamercolorbox}%
}


% Remove navigation symbols
\setbeamertemplate{navigation symbols}{}

% ---------------------------
% Packages
% ---------------------------
\usepackage{graphicx}
\usepackage{amsmath}
\usepackage{multicol}

% ---------------------------
% Aspect ratio selector
% ---------------------------

\newcommand{\choosebackground}{%
  \ifnum\myaspectratio=43
    \def\mybackground{background_43.png}%
  \else
    \ifnum\myaspectratio=169
      \def\mybackground{background_169.png}%
    \else
      \def\mybackground{background.png}% fallback
    \fi
  \fi
}
\choosebackground

\setbeamertemplate{background canvas}{%
  \includegraphics[width=\paperwidth]{\mybackground}%
}

% ---------------------------
% Title slide (no background, no numbering)
% ---------------------------
% usa se non hai ancora il pacchetto per i loghi social
% \usepackage{fontawesome5} 

\newcommand{\plainTitlePage}[2]{% #1 = immagine grande, #2 = lista social
  {%
    % background ghostwhite
    \setbeamertemplate{background canvas}[vertical shading][bottom=ghostwhite,top=ghostwhite]

    \begin{frame}[plain] % plain = no footer
      \vfill
      \begin{columns}[T,totalwidth=\textwidth]
        % Text column
        \column{0.6\textwidth}
          \raggedleft
          {\color{mainblue}\Huge \inserttitle\par}
          \bigskip\bigskip
          {\Large \insertauthor\par}
          \medskip
          {\large \insertdate\par}

          % Social links area
          \raggedleft
          {\small #2}

        % Space between columns
        \hspace{1cm}

        % Image column
        \column{0.35\textwidth}
          \centering
          \includegraphics[width=\linewidth]{#1}
      \end{columns}
      \vfill
    \end{frame}

    % restore background afterwards
    \setbeamertemplate{background canvas}{%
      \includegraphics[width=\paperwidth]{\mybackground}%
    }
  }%
}


% ---------------------------
% Frame numbering in bottom-right corner
% ---------------------------
\setbeamertemplate{footline}{%
  \leavevmode%
  \hfill%
  \usebeamerfont{footline}\usebeamercolor[fg]{footline}%
  {\tiny \insertframenumber{} / \inserttotalframenumber}%
  \hspace{0.5cm}\vspace{0.2cm}%
}
\setbeamerfont{footline}{size=\small}

% Información del documento
\title[Análisis Crítico: Sistemas Electromecánicos]{Informe Crítico sobre la Disciplina de Sistemas Electromecánicos}
\subtitle{Análisis Comparativo con Programas Internacionales}
\author{Dr. Vladímir Rodríguez Diez}
\institute{Universidad}
\date{\today}

% Configuración de notas
\setbeameroption{show notes}


\begin{document}
% Cheatsheet: https://www.ipgp.fr/~moguilny/LaTeX/fontawesome5Icons.pdf
\plainTitlePage{Portada1.png}{%
  % \faTwitter\ \href{https://twitter.com/vladimir12}{@myuser}\\
   % \faGithub\ \href{https://github.com/vladimir1284}{github.com/vladimir1284}\\
   % \faLinkedin\ \href{https://linkedin.com/in/vladímir-rodríguez-diez-12a76649}{/in/vladímir-rodríguez-diez}%
}

  % Índice
  \begin{frame}{Índice}
  \tableofcontents
  \note[item]{Estructura clara en 4 secciones principales}
  \note[item]{Primera parte: contexto y fundamentación}
  \note[item]{Segunda parte: contenidos actuales del programa}
  \note[item]{Tercera parte: comparación internacional (MIT)}
  \note[item]{Cuarta parte: conclusiones y recomendaciones}
  \end{frame}
  
  \section{Fundamentación de la Disciplina}
  
  \begin{frame}{Relevancia de los Sistemas Electromecánicos}
  \begin{block}{Contexto Energético Global}
  \begin{itemize}
  \item Más del 98\% de la energía eléctrica mundial requiere conversión electromecánica
  \item Alto porcentaje de consumo mediante motores eléctricos rotatorios
  \item Aplicaciones críticas: industria, servicios, transporte
  \end{itemize}
  \end{block}
  
  \pause
  
  \begin{block}{Núcleo de la Disciplina}
  \begin{itemize}
  \item Convertidores electromecánicos rotatorios (MER)
  \item Transformadores
  \item Nexo fundamental: sistema eléctrico - sistema mecánico
  \end{itemize}
  \end{block}
  
  % SUGERENCIA GRAFICA: Diagrama de flujo simple mostrando:
  % Energía Primaria -> Conversión Mecánica -> Conversión Eléctrica -> Distribución -> Motores -> Aplicaciones
  
  \note[item]{TIEMPO: 1.5 minutos}
  \note[item]{Enfatizar la universalidad: 98\% es un dato contundente}
  \note[item]{Conectar con desafíos actuales: eficiencia energética y sostenibilidad}
  \note[item]{Los convertidores electromecánicos son el corazón de la infraestructura energética moderna}
  \note[item]{Mencionar que esta relevancia justifica el análisis crítico que presentaremos}
  \end{frame}
  
  \begin{frame}{Imperativo Contemporáneo}
  \begin{alertblock}{Desafíos Globales}
  Problemas energéticos y ambientales requieren profesionales capacitados en:
  \begin{itemize}
  \item Proyección de sistemas eficientes
  \item Evaluación de rendimiento energético
  \item Mantenimiento predictivo
  \item Reducción de impacto ambiental
  \end{itemize}
  \end{alertblock}
  
  \pause
  
  \textbf{Implicación Pedagógica:} La formación en Sistemas Electromecánicos no es opcional, es estratégica para el futuro sostenible.
  
  \note[item]{TIEMPO: 1 minuto}
  \note[item]{Vincular con Agenda 2030 y objetivos de desarrollo sostenible}
  \note[item]{La disciplina tiene responsabilidad social: formar ingenieros para transición energética}
  \note[item]{Este contexto justifica un análisis crítico riguroso de nuestros programas}
  \end{frame}
  
  \section{Contenidos y Enfoque Actual}
  
  \begin{frame}{Estructura Curricular: Tres Pilares}
  \begin{columns}[T]
  \column{0.33\textwidth}
  \centering
  \textbf{Pilar 1}\\
  \textcolor{blue}{Magnetismo y Transformadores}
  \begin{itemize}
  \item Circuitos magnéticos
  \item Transformadores ideales y reales
  \item Configuraciones trifásicas
  \item Modelos matemáticos
  \end{itemize}
  
  \column{0.33\textwidth}
  \centering
  \textbf{Pilar 2}\\
  \textcolor{red}{Máquinas Eléctricas Rotatorias}
  \begin{itemize}
  \item Conversión energética
  \item Máquinas sincrónicas
  \item Máquinas asincrónicas
  \item Máquinas CC
  \end{itemize}
  
  \column{0.33\textwidth}
  \centering
  \textbf{Pilar 3}\\
  \textcolor{green!50!black}{Control y Electrónica}
  \begin{itemize}
  \item Teoría de control
  \item Accionamientos
  \item Variadores de velocidad
  \item Simulación
  \end{itemize}
  \end{columns}
  
  % SUGERENCIA GRAFICA: Tres columnas con iconos representativos para cada pilar
  
  \note[item]{TIEMPO: 1.5 minutos}
  \note[item]{Estructura coherente e integradora}
  \note[item]{Pilar 1: Base teórica fundamental}
  \note[item]{Pilar 2: Dispositivos y máquinas específicas}
  \note[item]{Pilar 3: Aplicación y control - conexión con industria 4.0}
  \note[item]{Esta estructura es sólida, pero evaluaremos si es suficiente comparada con estándares internacionales}
  \end{frame}
  
  \begin{frame}{Profundidad Temática: Transformadores}
  \textbf{Contenidos Específicos:}
  \begin{itemize}
  \item Características constructivas y principio de operación
  \item Circuito equivalente y ensayos normalizados
  \item Transformadores trifásicos: grupos de conexión
  \item Operación en paralelo y autotransformadores
  \item Modelado para régimen normal y transitorio
  \end{itemize}
  
  \vspace{0.3cm}
  
  \begin{exampleblock}{Fortaleza Identificada}
  Cobertura completa desde fundamentos físicos hasta aplicación industrial
  \end{exampleblock}
  
  \note[item]{TIEMPO: 1 minuto}
  \note[item]{Nivel de profundidad adecuado para ingeniería}
  \note[item]{Incluye tanto teoría como práctica}
  \note[item]{Los ensayos normalizados conectan con estándares internacionales}
  \note[item]{El modelado matemático permite simulación computacional}
  \end{frame}
  
  \begin{frame}{Profundidad Temática: MERs y Control}
  \begin{block}{Máquinas Eléctricas Rotatorias}
  \begin{itemize}
  \item Análisis energético y determinación de par
  \item Cuatro tipos principales: sincrónicas, asincrónicas, reluctancia conmutada, CC
  \item Énfasis especial en máquinas asincrónicas (industriales)
  \item Motores de alta eficiencia
  \end{itemize}
  \end{block}
  
  \begin{block}{Control e Ingeniería de Accionamiento}
  \begin{itemize}
  \item Lazo abierto y cerrado, funciones de transferencia
  \item Herramientas: MATLAB y SIMULINK
  \item Controladores: P, I, PI, PID
  \item Variadores de velocidad y control por orientación de campo
  \end{itemize}
  \end{block}
  
  \note[item]{TIEMPO: 1.5 minutos}
  \note[item]{Cobertura amplia de tipos de máquinas}
  \note[item]{Énfasis en asincrónicas justificado: son las más usadas industrialmente}
  \note[item]{Integración de control es un punto fuerte del programa}
  \note[item]{Uso de software profesional (MATLAB) prepara para la industria}
  \note[item]{Pregunta crítica a plantear después: ¿Es suficiente esta profundidad comparada internacionalmente?}
  \end{frame}
  
  \begin{frame}{Enfoque Pedagógico: Habilidades Prácticas}
  \textbf{Objetivos de Formación:}
  \begin{enumerate}
  \item Modelado matemático de transformadores y MERs
  \item Simulación de comportamiento en régimen normal y transitorio
  \item Conexión y puesta en marcha de máquinas
  \item Análisis y síntesis de sistemas de control
  \item Selección y ajuste de variadores de velocidad
  \end{enumerate}
  
  \vspace{0.3cm}
  
  \begin{alertblock}{Evaluación Rigurosa}
  Examen de prácticas de laboratorio: prueba eliminatoria (nota mínima 5/10). Procedimientos basados en normas internacionales vigentes.
  \end{alertblock}
  
  \note[item]{TIEMPO: 1 minuto}
  \note[item]{Orientación práctica clara y bien definida}
  \note[item]{El examen eliminatorio de laboratorio garantiza competencias prácticas}
  \note[item]{Uso de normas internacionales es positivo}
  \note[item]{Balance teoría-práctica parece adecuado}
  \note[item]{Preparación: ¿Esto es estándar internacional o excepcional?}
  \end{frame}
  
  \section{Comparación con Estándares Internacionales (MIT)}
  
  \begin{frame}{MIT: Enfoque Diferencial}
  \begin{block}{Análisis Comparativo}
  Se analizaron tres cursos del MIT:
  \begin{itemize}
  \item 6.007 (Pregrado): Electromagnetic Energy
  \item 6.302 (Posgrado): Feedback Systems
  \item 6.245 (Posgrado): Multivariable Control Systems
  \end{itemize}
  \end{block}
  
  \pause
  
  \textbf{Diferencias identificadas:}
  \begin{enumerate}
  \item Integración de física cuántica y optoelectrónica
  \item Especialización avanzada en control
  \item Segregación de contenidos por nivel
  \end{enumerate}
  
  % SUGERENCIA GRAFICA: Tabla comparativa simple con tres columnas:
  % Aspecto | Programa Analizado | MIT
  
  \note[item]{TIEMPO: 1 minuto}
  \note[item]{MIT representa excelencia internacional reconocida}
  \note[item]{No buscamos copiar, sino identificar brechas y oportunidades}
  \note[item]{Tres diferencias principales que exploraremos}
  \note[item]{Importante: diferencias no implican deficiencias, sino enfoques distintos}
  \end{frame}
  
  \begin{frame}{Diferencia 1: Física Fundamental Ampliada}
  \begin{columns}
  \column{0.5\textwidth}
  \textbf{Programa Analizado}
  \begin{itemize}
  \item Conceptos básicos de electricidad y magnetismo
  \item Enfoque en conversión electromecánica clásica
  \item Generación y utilización a gran escala
  \end{itemize}
  
  \column{0.6\textwidth}
  \textbf{MIT 6.007}
  \begin{itemize}
  \item Electromagnetismo clásico \textbf{y cuántico}
  \item Fotones como ondas y partículas
  \item Dispositivos modernos: células solares, láseres, pantallas
  \item Escala macroscópica \textbf{y cuántica}
  \end{itemize}
  \end{columns}
  
  \vspace{0.5cm}
  
  \begin{alertblock}{Observación Crítica}
  El programa del MIT integra tecnologías emergentes optoelectrónicas ausentes en nuestro currículo.
  \end{alertblock}
  
  \note[item]{TIEMPO: 1.5 minutos}
  \note[item]{Esta es una diferencia significativa}
  \note[item]{MIT amplía el concepto de "conversión de energía electromagnética"}
  \note[item]{Incluye tecnologías del siglo XXI: solar, LED, láser}
  \note[item]{Pregunta crítica: ¿Deberíamos incorporar estos contenidos?}
  \note[item]{Consideración: nuestro enfoque es más tradicional pero sólido en su ámbito}
  \note[item]{Posible área de expansión curricular}
  \end{frame}
  
  \begin{frame}{Diferencia 2: Control Especializado vs. Integrado}
  \textbf{Nuestro Enfoque:}
  \begin{itemize}
    \small
    \item Control integrado en la disciplina general
    \item Controladores P, I, PI, PID
    \item Variadores de velocidad
    \item Suficiente para operación y mantenimiento industrial
  \end{itemize}
  
  \pause
  
  \textbf{Enfoque MIT:}
  \begin{itemize}
    \small
    \item Cursos de posgrado dedicados exclusivamente a control
    \item 6.302: Sistemas de retroalimentación con laboratorios especializados
    \item 6.245: Metodologías avanzadas - optimización H-infinity, Mu Synthesis, LMI
    \item Diseño multivariable asistido por computadora
  \end{itemize}
  
  \pause
  {\small 
    \textbf{Interpretación:} integración vs. especialización profunda.
  }

  \note[item]{TIEMPO: 2 minutos}
  \note[item]{Aquí no hay un "mejor" absoluto, depende de objetivos}
  \note[item]{Nuestro enfoque: formar ingenieros operativos}
  \note[item]{Enfoque MIT: formar también investigadores en control avanzado}
  \note[item]{H-infinity y Mu Synthesis son técnicas de frontera en control robusto}
  \note[item]{Pregunta: ¿Nuestros graduados necesitan este nivel en pregrado?}
  \note[item]{Posible solución: mantener integración en pregrado, ofrecer especialización en posgrado}
  \end{frame}
  
  \begin{frame}{Diferencia 3: Amplitud vs. Profundidad}
  % SUGERENCIA GRAFICA: Diagrama de Venn o esquema visual comparando alcances
  
  \begin{block}{Programa Analizado}
  \textbf{Enfoque de Amplitud Aplicada}
  \begin{itemize}
  \item Cobertura integral de conversión electromecánica tradicional
  \item Orientación a aplicaciones industriales inmediatas
  \item Formación de ingenieros para infraestructura energética actual
  \end{itemize}
  \end{block}
  
  \begin{block}{MIT}
  \textbf{Enfoque de Profundidad Fundamental}
  \begin{itemize}
  \item Fundamentos físicos extendidos (cuántica)
  \item Segregación por especialización (cursos dedicados)
  \item Orientación hacia investigación y desarrollo tecnológico
  \end{itemize}
  \end{block}
  
  \note[item]{TIEMPO: 1 minuto}
  \note[item]{Dos modelos educativos legítimos}
  \note[item]{Nuestro modelo: generalista aplicado}
  \note[item]{Modelo MIT: especialista investigador}
  \note[item]{Ambos producen ingenieros competentes pero con perfiles diferentes}
  \note[item]{La pregunta es: ¿cuál necesita nuestra industria y sociedad?}
  \end{frame}
  
  \section{Análisis Crítico y Conclusiones}
  
  \begin{frame}{Fortalezas del Programa Actual}
  \begin{enumerate}
  \item \textbf{Integración coherente:} Tres pilares bien articulados
  \item \textbf{Orientación práctica:} Laboratorios con evaluación rigurosa
  \item \textbf{Cobertura completa:} De fundamentos a aplicación industrial
  \item \textbf{Herramientas modernas:} MATLAB, SIMULINK
  \item \textbf{Normas internacionales:} Procedimientos estandarizados
  \item \textbf{Relevancia actual:} Responde a demandas de eficiencia energética
  \end{enumerate}
  
  \vspace{0.3cm}
  
  \begin{exampleblock}{Valoración}
  El programa forma ingenieros capacitados para enfrentar los desafíos actuales de la infraestructura energética.
  \end{exampleblock}
  
  \note[item]{TIEMPO: 1.5 minutos}
  \note[item]{Importante reconocer fortalezas antes de críticas}
  \note[item]{El programa tiene bases sólidas}
  \note[item]{Cumple su objetivo fundamental: formar ingenieros operativos}
  \note[item]{La integración de tres pilares es pedagogicamente efectiva}
  \note[item]{No todo debe cambiar - hay mucho que preservar}
  \end{frame}
  
  \begin{frame}{Áreas de Oportunidad Identificadas}
  \textbf{1. Tecnologías Emergentes}
  \begin{itemize}
  \item Incorporación limitada de dispositivos optoelectrónicos
  \item Ausencia de contenidos sobre energías renovables modernas
  \item Brecha en tecnologías de escala cuántica
  \end{itemize}
  
  \pause
  
  \textbf{2. Profundización en Control Avanzado}
  \begin{itemize}
  \item Control multivariable limitado a nivel introductorio
  \item Técnicas de optimización avanzadas ausentes en pregrado
  \item Oportunidad para cursos de posgrado especializados
  \end{itemize}
  
  \pause
  
  \textbf{3. Actualización Metodológica}
  \begin{itemize}
  \item Posible integración de metodologías de diseño asistido por IA
  \item Mayor énfasis en simulación de sistemas complejos
  \end{itemize}
  
  \note[item]{TIEMPO: 2 minutos}
  \note[item]{Crítica constructiva, no destructiva}
  \note[item]{Área 1: El mundo va hacia renovables, vehículos eléctricos, electrónica de potencia avanzada}
  \note[item]{Células solares, LED, sistemas de almacenamiento son conversión electromagnética moderna}
  \note[item]{Área 2: Control avanzado podría ser posgrado, no necesariamente pregrado}
  \note[item]{Área 3: IA y machine learning están transformando el diseño de sistemas de control}
  \note[item]{Estas son oportunidades, no deficiencias críticas}
  \end{frame}
  
  \begin{frame}{Recomendaciones Estratégicas}
  \begin{block}{Corto Plazo}
  \begin{itemize}
    \small
  \item Incorporar módulo sobre conversión fotovoltaica y LED
  \item Ampliar contenido sobre vehículos eléctricos y almacenamiento
  \item Actualizar casos de estudio con aplicaciones contemporáneas
  \end{itemize}
  \end{block}

  \vspace{-0.3cm}

  \begin{block}{Mediano Plazo}
  \begin{itemize}
    \small
  \item Desarrollar curso optativo de posgrado en control multivariable
  \item Establecer laboratorio de sistemas optoelectrónicos
  \item Colaboraciones internacionales para actualización docente
  \end{itemize}
  \end{block}
  
  \vspace{-0.3cm}

  \begin{block}{Largo Plazo}
  \begin{itemize}
    \small
  \item Rediseño curricular integrando física cuántica aplicada
  \item Programa de posgrado especializado en control avanzado
  \end{itemize}
  \end{block}
  
  \note[item]{TIEMPO: 2 minutos}
  \note[item]{Recomendaciones realistas y escalonadas}
  \note[item]{Corto plazo: ajustes sin reestructuración mayor}
  \note[item]{Mediano plazo: expansión gradual}
  \note[item]{Largo plazo: visión transformadora}
  \note[item]{Importante: mantener fortalezas actuales mientras se innova}
  \note[item]{No todo debe hacerse simultáneamente - priorizar según recursos}
  \end{frame}
  
  \begin{frame}{Conclusión General}
  \begin{alertblock}{Síntesis}
  La disciplina de Sistemas Electromecánicos es \textbf{fundamental, bien estructurada y cumple su misión} de formar ingenieros para la infraestructura energética actual.
  \end{alertblock}
  
  \pause

  \vspace{-0.2cm}

  \textbf{Balance Final:}
  \begin{itemize}
  \item[$+$] Sólida formación en conversión electromecánica clásica
  \item[$+$] Integración efectiva de teoría, práctica y control
  \item[$+$] Preparación adecuada para industria tradicional
  \item[$-$] Limitada incorporación de tecnologías emergentes
  \item[$-$] Control avanzado solo a nivel introductorio
  \end{itemize}
  
  % \vspace{0.3cm}
  
  \textbf{Visión:} Evolucionar sin perder identidad - expandir hacia tecnologías del siglo XXI manteniendo fortalezas consolidadas.
  
  \note[item]{TIEMPO: 1.5 minutos}
  \note[item]{Mensaje balanceado: fortalezas y oportunidades}
  \note[item]{No hay crisis, hay oportunidad de mejora}
  \note[item]{El programa actual es competente, puede ser excepcional}
  \note[item]{La comparación con MIT no es para copiarlo, sino para inspirar evolución}
  \note[item]{Preparar para transición: estamos bien, podemos estar mejor}
  \end{frame}
  
  \begin{frame}{Impacto de las Mejoras Propuestas}
  % SUGERENCIA GRAFICA: Diagrama radial o matriz mostrando impactos
  
  \textbf{Beneficios Esperados:}
  \begin{enumerate}
  \item \textbf{Para los estudiantes:} Formación más competitiva internacionalmente
  \item \textbf{Para la institución:} Mayor prestigio y atracción de talento
  \item \textbf{Para la industria:} Ingenieros preparados para transición energética
  \item \textbf{Para la sociedad:} Contribución a sostenibilidad y innovación
  \end{enumerate}
  
  \vspace{0.5cm}
  
  \begin{exampleblock}{Llamado a la Acción}
  La actualización curricular no es solo deseable, es necesaria para mantener la relevancia de la disciplina en un mundo tecnológicamente dinámico.
  \end{exampleblock}
  
  \note[item]{TIEMPO: 1 minuto}
  \note[item]{Conectar mejoras con stakeholders múltiples}
  \note[item]{No es capricho académico, es necesidad estratégica}
  \note[item]{El mundo de la energía está cambiando rápidamente}
  \note[item]{Nuestros graduados competirán en mercado global}
  \note[item]{La disciplina debe evolucionar para mantenerse relevante}
  \end{frame}
  
  \begin{frame}{Referencias Principales}
  \small
  \textbf{Programas Analizados:}
  \begin{itemize}
  \item Circuitos Magnéticos y Transformadores: Programa
  \item Máquinas Eléctricas (2011) - Escuela Universitaria de Ingeniería Técnica Minera
  \item Plan de Estudios: Disciplina Sistemas Electromecánicos
  \end{itemize}
  
  \vspace{0.3cm}
  
  \textbf{Referencias Internacionales (MIT OpenCourseWare):}
  \begin{itemize}
  \item 6.007: Electromagnetic Energy - From Motors to Lasers
  \item 6.302: Feedback Systems
  \item 6.245: Multivariable Control Systems
  \end{itemize}
  
  \note[item]{TIEMPO: 30 segundos}
  \note[item]{Todas las fuentes son documentadas y verificables}
  \note[item]{Análisis basado en documentación oficial}
  \note[item]{MIT OpenCourseWare es recurso público y reconocido}
  \end{frame}
  
  \begin{frame}[plain]
  \begin{center}
  \Huge Gracias por su atención
  
  \vspace{1cm}
  
  \Large Abierto a preguntas y comentarios del tribunal
  \end{center}
  
  \note[item]{TIEMPO: Variable según preguntas}
  \note[item]{Agradecer al tribunal su tiempo y atención}
  \note[item]{Mostrar disposición constructiva ante críticas}
  \note[item]{Preparación para preguntas probables:}
  \note[item]{- ¿Por qué MIT como referencia única?}
  \note[item]{- ¿Costos de implementación de mejoras?}
  \note[item]{- ¿Priorización de recomendaciones?}
  \note[item]{- ¿Experiencia docente propia en la disciplina?}
  \note[item]{Mantener postura: análisis crítico constructivo, no destructivo}
  \end{frame}
  
  \end{document}